\chapter{Conclusions}

Stearoyl-CoA desaturase (SCD1) plays a critical role in the metabolism of fatty acids and is an attractive target for the development of new drugs. It belongs to the large and poorly understood group of transmembrane non-heme diiron enzymes. Its structure was determined in 2015 and since then only one paper investigated its reaction mechanism, utilizing a DFT cluster model. The DFT cluster model study proposed a potential reactive intermediate, referred to as intermediate A in this thesis. Our goal was to investigate its dynamical properties both with classical and multiscale MD simulations.

Our classical MD simulations of intermediate A suggest that it is not able to form a favourable reactive complex with stearoyl-CoA which would be consistent with the experimentally determined selectivity of SCD1. The results implied the importance of a conserved asparagine residue in controlling the conformation of the substrate. Based on this observation, we proposed a new intermediate B in which a water ligand molecule occupies a position where it stabilizes the conserved asparagine residue. Intermediate B only slightly differs from intermediate A and its formation should be possible within the context of the already proposed mechanism. Notably, the results of classical MD simulations suggest that intermediate B, in contrast to intermediate A, is able to form a favourable complex with the substrate. The structure of intermediate B also appears to be more consistent with the presence of two iron ions due to the H-bond between the oxo group on Fe\textsubscript{B} and the hydroxide on Fe\textsubscript{A}. This H-bond implies a greater degree of cooperation between the two centers compared to intermediate A, where the Fe\textsubscript{A} center only plays a passive role during the initial HAA.

We investigated the reactivity of intermediate B with QM/MM methods, using the efficient semiempirical GFN2-xTB method in combination with B3LYP energy corrections. Our best estimate for the free energy barrier of the first HAA step on C$_{9}$ is 16.9 kcal/mol. Based on the Eyring equation, the corresponding rate constant is 3 s$^{-1}$. This value is slightly lower than for most enzymes, but there are numerous ways to improve this estimate, such as considering more pathways, employing methods beyond B3LYP, incorporating zero-point energy and tunneling corrections and increasing sampling. Our objective was not to provide a very accurate estimate of the free energy barrier, but rather to demonstrate the potential reactivity of intermediate B, which we accomplished.

While GFN2-xTB demonstrated the ability to predicted reasonable structures, it encountered certain challenges during MD simulations. Nevertheless, it performed surprisingly well when taking into account the complexity of the system. Given its computational efficiency, it certainly deserves attention in future studies.

Potential next research steps could be to conduct a cluster model or QM/MM study of the HAA step by intermediate B using a more accurate method. The resulting structures and barriers can be compared to our results obtained with GFN2-xTB. It would be beneficial to consider different pathways and spin states. We suggest that all future cluster model studies include the conserved asparagine residue in their models. 

It would also be interesting to investigate the oxygen activation mechanism of SCD1 using QM/MM. The current opinion is that the mechanism does not involve the formation of bridged peroxo- and oxo-complexes because of the large iron-iron distance. However, spectroscopic data from another enzyme within the same family, alkane $\omega$-hydroxylase (AlkB), suggests otherwise. The crystal structure of AlkB was determined recently, so it would also be an excellent candidate for this study.

In conclusion, our findings provide valuable insight into the mechanism of SCD1, which can have implications for the design of new inhibitors and advance the understanding of other transmembrane non-heme diiron enzymes. 

