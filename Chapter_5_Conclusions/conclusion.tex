\chapter{Conclusions and outlook}

\subsubsection{Final thoughts}
This thesis aimed to investigate the hydrolysis mechanism of methyl diphosphate in water, with a particular focus on the influence of protonation state and solvent environment, by leveraging state-of-the-art neural network potentials and enhanced sampling techniques. The work demonstrates that, by combining data-efficient equivariant graph neural networks (specifically, NequIP) with well-tempered metadynamics, it is possible to obtain detailed free energy surfaces and mechanistic insights for complex, reactive systems at a fraction of the computational cost of traditional \textit{ab initio} molecular dynamics (AIMD).

A comprehensive and diverse dataset was constructed through an iterative learning loop, employing density-aware sampling to ensure balanced coverage of the relevant collective variable space. The final dataset, comprising 12,000 training/validation and 1,800 test configurations, enabled the training of highly accurate \acp{nnp}. The best-performing model, with tensor rank $\ell=1$, achieved force root-mean-square errors (RMSE) of 37.069 meV/\AA\ and energy mean absolute errors (MAE) below 0.3 meV/atom on the test set - well within the range considered a very good fit by current community standards. Notably, this level of accuracy was achieved with a relatively modest dataset size, highlighting the data efficiency of equivariant GNNs.

The NNP-driven AIMD simulations showed excellent stability over nanosecond \; timescales, with no evidence of numerical instabilities or unphysical artefacts. The structural properties of water, as assessed by the oxygen-oxygen radial distribution function (RDF), were in qualitative agreement with experimental data and previous computational studies at the same level of theory, albeit with some over-structuring typical for the PBE-D3 level of theory at ambient temperature.

For the first time, the nanosecond long sampling of the reaction space was performed which sets ground for the accurate and detailed description of the free energy landscape. The free energy surfaces obtained for both methyl diphosphate trianion (MeDP) and dianion (MeHDP) revealed the presence of both associative and dissociative reaction pathways. For MeDP, the dissociative/concerted (D\textsubscript{N}A\textsubscript{N}) pathway was found to be energetically preferred, with a barrier height of 28.22~kcal/mol, in excellent agreement with the experimental value of 29.2~kcal/mol. The associative pathway was also accessible but featured a significantly higher barrier. In contrast, for MeHDP, the associative/concerted (A\textsubscript{N}D\textsubscript{N}) pathway was better sampled and displayed a lower barrier (30.43~kcal/mol) than the corresponding pathway in MeDP, consistent with the known effect of protonation in enhancing reactivity. The dissociative pathway for MeHDP appeared undersampled, suggesting the need for longer simulations to fully resolve its role.

The analysis of the CV evolution and the observation of multiple recrossings between reactant and product states provided further evidence for the quality of sampling and the reliability of the obtained FES, although full convergence, especially in the transition state regions, remains a challenge. The proton transfer (PT) mechanism was found to involve both 1 and 3 water-mediated pathways, with spontaneous PT events observed during the dynamics, underscoring the ability of the NNP to capture these extremely fast occuring events.

Despite these successes, several caveats must be acknowledged. The accuracy of the NNP is ultimately limited by the quality of the underlying reference data (PBE-D3(BJ)/TZV2P), which is known to over-structure water and may underestimate or overestimate certain reaction barriers. The FES, while showing clear signs of convergence, is not fully converged in the transition state and product regions, particularly for MeHDP. Furthermore, the absence of explicit validation of transition state structures (e.g., via normal mode analysis) means that the precise character of the saddle points remains to be confirmed.

\subsubsection{Future directions}
The results presented in this thesis open several avenues for future research and methodological improvement.

\begin{itemize}
    \item[--] \textit{Extended sampling and FES convergence:} Achieving fully converged free energy surfaces will require longer metadynamics simulations (potentially more than 6 ns, as seen in related studies). Employing multiple-walker metadynamics could accelerate convergence.
    
    \item[--] \textit{Higher-level reference data:} The accuracy of the NNP could be further improved by retraining on reference data generated at a higher level of theory (e.g., hybrid functionals).
    
    \item[--] \textit{Transition state validation:} To unambiguously characterise the nature of the transition states, normal mode analysis should be performed on candidate structures extracted from the FES, confirming the presence of a single imaginary frequency.
      
    \item[--] \textit{Effect of enthalpy and entropy:} In order to see how the enthalpy and entropy contribute to the overall reaction barrier, it would be advantageous to perform biased simulations at different temperatures and get by means of linear fit the enthalpic and entropic contributions from the $\Delta G^{\ddagger} = \Delta H^{\ddagger} - T \Delta S^{\ddagger}$ relation. This would provide a more complete picture of the reaction mechanism.
   
    \item[--] \textit{Explicit treatment of \ac{pt}:} The spontaneous \ac{pt} events observed here suggest that the NNP is capable of capturing proton dynamics, but a more systematic investigation, potentially using dedicated CVs for PT and enhanced sampling along these coordinates, would provide deeper insight into the PT mechanism.
    
    \item[--] \textit{Extension to more complex systems:} The workflow developed here can be readily extended to study the hydrolysis of more complex phosphate esters (e.g., ADP, ATP, GTP) and to include the effects of metal ions (such as Mg$^{2+}$). Such studies would bridge the gap between model systems and biological reality.
\end{itemize}

In summary, this thesis demonstrates the feasibility and power of combining machine learning interatomic potentials with enhanced sampling to unravel complex reaction mechanisms in solution. While challenges remain, particularly in achieving full FES convergence and in addressing the limitations of the underlying electronic structure methods, the approach outlined here provides a practical and scalable framework for future studies of chemical reactivity in realistic environments. As neural network potentials continue to improve, they are proned to become an indispensable tool in the computational chemist's toolkit, enabling the exploration of chemical space with unprecedented accuracy and efficiency.