\chapter*{Summary}
Stearoyl-CoA desaturase (SCD1) plays an important role in the metabolism of fatty acids and is a promising therapeutic target. However, the underlying mechanism of SCD1, as well as other transmembrane non-heme diiron enzymes, remains poorly understood. This study builds upon a previous DFT cluster model study which proposed a potential reactive intermediate of SCD1. We assessed its dynamical properties by employing classical and multiscale molecular dynamics (MD) simulations. Our classical MD simulations revealed that the proposed intermediate lacks the ability to form a favourable reactive complex with stearoyl-CoA, highlighting the significance of a conserved asparagine residue in controlling the substrate's orientation. Motivated by these observations, we proposed a new intermediate in which a water molecule is strategically placed to stabilize the conserved asparagine residue. Subsequent classical MD simulations showed that the new intermediate is able to form a reactive complex with the substrate, consistent with the experimentally observed selectivity of SCD1. The free energy barrier for the first hydrogen atom abstraction (HAA) step on C$_{9}$ by the new intermediate was estimated to be 16.9 kcal/mol. The estimate is based on a hybrid quantum mechanics/molecular mechanics (QM/MM) approach utilizing the efficient semiempirical GFN2-xTB method in combination with B3LYP energy corrections. Considering its computational efficiency, GFN2-xTB seems to be a promising tool for the study of complex transition metal systems. Overall, our findings provide valuable insights into the mechanism of SCD1, thereby advancing the understanding of the entire class of transmembrane non-heme diiron enzymes. Furthermore, the findings can potentially help in the design of new inhibitors.

%\chapter*{Summary accessible to the broad public}
%Enzymes are important proteins that accelerate chemical reactions in our bodies. They typically achieve this by reducing the energy required for a reaction to occur. However, in many cases, we do not fully understand the details of how enzymes accomplish this. Experimental studies can be very challenging, so it is common to employ computer simulations to help us understand how enzymes work. Computer simulations try to mimic the atomic behaviour of the real system using some approximate models based on classical and/or quantum mechanics. The enzyme of interest in this thesis is stearoyl-CoA desaturase (SCD1). It plays a crucial role in the metabolism of fatty acids and has been linked to increased cancer growth. Studies found that its inhibition can slow down cancer growth, making it a promising therapeutic target. This thesis builds upon a previous study which suggested a potential reactive intermediate involved in the reactivity of SCD1. We used advanced computer simulations to investigate the proposed intermediate in more detail. The simulations showed that the proposed intermediate lacked the ability to interact favourably with its target molecule. Based on these results, we proposed a new intermediate that should better stabilize the rest of the protein structure and make it more favourable. Subsequent simulations indeed showed that the new intermediate is able to combine with its target molecule in a manner consistent with experimental observations. Additionally, we estimated the energy needed for the reaction to occur and found that it falls in the range typical for other enzymes, further supporting our proposal. Our findings advance the understanding of SCD1 and related enzymes, potentially helping in the design of new drugs targeting SCD1.