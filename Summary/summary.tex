\chapter*{Summary}
Phosphates are fundamental to life, underpinning the storage and transfer of energy and the stability of genetic material in all living organisms. Despite their ubiquity, the detailed mechanisms by which phosphate-containing molecules undergo hydrolysis in aqueous environments remain a subject of debate and intense research, owing to the complexity of the underlying chemistry and the limitations of computational approaches.

This thesis addresses the hydrolysis mechanism of methyl diphosphate in water, a model system for biologically relevant phosphate esters, by combining advanced machine learning techniques with enhanced sampling molecular simulations. Specifically, a neural network potential (NNP) based on the NequIP equivariant graph neural network architecture was trained on a carefully curated dataset of quantum mechanical reference calculations. This approach enabled the efficient and accurate exploration of the reaction's free energy landscape at a level of detail and timescale that would be too expensive using traditional \textit{ab initio} molecular dynamics.

For the first time in the phosphate hydrolysis related research, the nanosecond long exploration of the free energy landscape of the reaction was performed. The NNP-driven simulations revealed both associative and dissociative reaction pathways. For the trianion (MeDP), the dissociative/concerted D\textsubscript{N}A\textsubscript{N} pathway was energetically preferred, with a computed barrier height in excellent agreement with experimental data. In contrast, for the protonated dianion (MeHDP), the associative/concerted A\textsubscript{N}D\textsubscript{N} pathway was better sampled, and the barrier was lower than for the same mechanism in case of MeDP, reflecting the known effect of protonation in enhancing reactivity. The simulations also captured spontaneous proton transfer events, highlighting the ability of the NNP to model fast proton dynamics in solution.

A key strength of this work lies in the data-efficient training of the NNP, which achieved high accuracy with a relatively small dataset, and in the stability of the resulting molecular dynamics simulations over nanosecond timescales. The structural properties of water and the convergence behaviour of the free energy surfaces were carefully benchmarked against experimental and computational standards, lending confidence to the mechanistic insights obtained.

% Overall, this thesis demonstrates the power and practicality of combining machine learning potentials with enhanced sampling to study complex chemical reactions in realistic environments. The methodologies developed here are readily transferable to more complex systems, including biologically relevant phosphates and enzymatic environments, and point towards a future where artificial intelligence plays a central role in advancing our understanding of chemical reactivity at the molecular level.
