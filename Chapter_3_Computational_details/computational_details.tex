\chapter{Computational Details}
% This chapter gives the computational details of all methods used in the thesis. The first section describes the system preparation and setup of classical molecular dynamics simulations. The MD simulations of the proposed reactive intermediate Int4$_{\text{A}-\alpha}$ suggest that it has unfavorable dynamical properties, so simulations of a newly proposed intermediate were also performed. From now on Int4$_{\text{A}-\alpha}$ is referred to as intermediate A while our newly proposed intermediate is referred to as intermediate B (Fig.\,\ref{fig:IntA_IntB}). They differ in the position of the oxo group on Fe$_{\text{B}}$. The second section gives details on the cluster model relaxed potential energy surface (PES) scan with GFN2-xTB \cite{Bannwarth2019,Grimme2017} for the HAA step by A which was used to assess the applicability of the GFN2-xTB method. The last section describes how QM/MM PES scans and MD simulations were used to study the reactivity of the newly proposed intermediate B.

\section{Training dataset generation}

\subsection{System preparation}

\subsection{Initial equillibration using the classifal forcefields}

\subsection{xTB based exploration of the configuration space}

\subsection{Data labeling}

\subsection{Iterative training of the neural network potential}

\subsubsection{First round}

\subsubsection{Second round}

\subsubsection{Third round}



\section{Production runs at different temperatures}



\section{Validation of the transition states}



\section{Data analysis and visualisation}