\chapter{Computational Details}
This chapter provides the details of the computational methods used in this work. The first section describes the generation of the training dataset, including the preparation of the system, initial equilibration using the molecular mechanics, exploration of the configuration space at the xTB level, further data labeling, and iterative training of the neural network potential. The second section discusses the production runs at different temperatures using the fitted neural network potential. The third section describes the workflow of validating the transition states obtained from the simulations following the partial Hessian formalism. Finally, the fourth section presents the data analysis and visualisation techniques employed to interpret the results.

\section{Training dataset generation}

\subsection{System preparation}
The systems were prepared using the \texttt{CHARMM-GUI} webserver's functionality \citep{jo_charmm-gui_2008}. In particular, the \texttt{Multicomponent Assembler} interface \citep{kern_charmm-gui_2024} was utilised. 

As a first step, the singly protonated and deprotonated forms of the methyl diphosphate were parametrised in \texttt{CGenFF} \citep{kim_charmm-gui_2017}, i.e. CHARMM General Forcefield. These states of the methyl diphosphate were chosen based on the fact that pyrophosphoric (diphosphoric) acid has the following dissociation constants \citep{haynes_crc_2016}:
\begin{align*}
    \mathrm{H_4P_2O_7} \rightleftharpoons \mathrm{[H_3P_2O_7]^-} + \mathrm{H^+},\quad \mathrm{p}K_\mathrm{a} = 0.91 \\
    \mathrm{[H_3P_2O_7]^-} \rightleftharpoons \mathrm{[H_2P_2O_7]^{2-}} + \mathrm{H^+},\quad \mathrm{p}K_\mathrm{a} = 2.10 \\
    \mathrm{[H_2P_2O_7]^{2-}} \rightleftharpoons \mathrm{[HP_2O_7]^{3-}} + \mathrm{H^+},\quad \mathrm{p}K_\mathrm{a} = 6.70 \\
    \mathrm{[HP_2O_7]^{3-}} \rightleftharpoons \mathrm{[P_2O_7]^{4-}} + \mathrm{H^+},\quad \mathrm{p}K_\mathrm{a} = 9.32
\end{align*}
Thus, at the physiological pH of 7.4 this acid exists as an equillibrium between the doubly and singly protonated forms. As an assumption, the methyl group can be considered as a proton, therefore we condsidered the methyl diphosphate molecule to exist as a mixture of the singly (MeHDP) and deprotonated (MeDP) forms at the physiological pH.

After succesfully parametrising the molecules, the system was solvated in a cubic box of water molecules together with the sodium counterions Na\textsuperscript{+} to neutralise the charge. The final system composition can be seen in Table \ref{tab:system-before-equillibration}.

\subsection{Initial equillibration using the classical forcefields}

\begin{table}[htbp]
    \centering
    \caption{System composition and simulation box details. \textsuperscript{1}The final dimensions were obtained after the NPT run using the CHARMM36m forcefield.}
    \label{tab:system-before-equillibration}
    \begin{tabular}{@{}lccc@{}}
    \toprule
    System & Equillibrated box dimensions\textsuperscript{1} (\AA) & No. of water molecules & No. of Na\textsuperscript{+} \\
    \midrule
    MeDP  & $15.877 \times 15.877 \times 15.877$ & 119 & 3 \\
    MeHDP & $15.901 \times 15.901 \times 15.901$ & 124 & 2 \\
    \bottomrule
    \end{tabular}
\end{table}
\subsection{xTB based exploration of the configuration space}

\subsection{Data labeling}

\subsection{Iterative training of the neural network potential}

\subsubsection{First round}

\subsubsection{Second round}

\subsubsection{Third round}



\section{Production runs at different temperatures}



\section{Validation of the transition states}



\section{Data analysis and visualisation}