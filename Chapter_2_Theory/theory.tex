\chapter{Theory}
% This chapter provides the theoretical background of methods used to study the selectivity and reactivity of SCD1. Firstly, classical molecular mechanics (MM) force fields and their development are explained. It is also possible to describe a part of the system on the quantum mechanical (QM) level, which is explained in the second section. Section three describes how the developed energy expressions can be used to run molecular dynamics (MD) simulations. Finally, section four describes the calculation of various properties from MD simulations with emphasis on free energies. The material in sections 2.1 and 2.3 is mostly based on ``Introduction to Computational Chemistry'' by Frank Jensen \cite{Jensen2007} and in section 2.2 on an excellent review article by Hans Martin Senn and Walter Thiel \cite{Senn2009}. The material in section 2.4 is based on the textbook ``Molecular Modelling (Principles and Applications)'' by Andrew R. Leach \cite{Leach}.

\section{A brief introduction to statistical mechanics}

\subsection{Classical forcefields and molecular dynamics}

\subsection{The canonical ensemble and free energy calculations}

\subsection{Free energy techniques}



\section{Transition state theory}



\section{Density functional theory}

\subsection{The Kohn-Sham approach}

\subsection{Generalised gradient approximation and PBE functional}

\subsection{\textit{Ab initio} molecular dynamics and GPW method}



\section{Extended tight binding}



\section{Neural network potentials}

\subsection{Deep neural networks}

\subsubsection{Multilayer perceptron}

\subsubsection{Graph neural networks}

\subsubsection{Message passing neural networks}

\subsection{Invariance and equivariance}

\subsection{Behler-Parrinello neural network potentials}

\subsection{Equivariant neural network potentials}